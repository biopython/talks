\documentclass[10pt,oneside]{article}

%%%%%%%%%%%%%
\setlength{\textheight}{8.75in} %Letter is 11in, less 2 for margins, less 0.25 for footer
\setlength{\oddsidemargin}{0.0in} %gets +1inc
\setlength{\evensidemargin}{0.0in} %gets +1inch
\setlength{\textwidth}{6.50in} %Letter is 8.5, less 2 inches for margins
\setlength{\topmargin}{0.5in}
\setlength{\headheight}{0in}
\setlength{\headsep}{0in}
\setlength{\parindent}{0.25in}
%%%%%%%%%%%%

% use letters instead of symbols to accommodate >7 authors
\makeatletter
\let\@fnsymbol\@alph
\makeatother

\usepackage[utf8]{inputenc}
\usepackage[numbers]{natbib}
\usepackage{graphicx}
\usepackage[colorlinks=true,citecolor=black,urlcolor=blue]{hyperref}

\title{%Hack to get the logo on the PDF front page:
\vspace{-1.5in}
\includegraphics[width=0.3\textwidth]{biopython_logo_s.png} \\
\vspace{3mm}Biopython Project Update 2020}
\author{
    Peter Cock\thanks{Information and Computational Sciences, James Hutton Institute, Invergowrie, Dundee, UK},\\
    and the Biopython Contributors\thanks{See \href{https://github.com/biopython/biopython/blob/master/CONTRIB.rst}{contributor listing on GitHub}.}}
\date{Bioinformatics Open Source Conference (BOSC),\\Bioinformatics Community Conference 2020}

\begin{document}
\maketitle
\thispagestyle{empty}

\vspace{-0.2in}
\noindent
Website: \url{http://biopython.org} \\
Repository: \url{https://github.com/biopython/biopython} \\
License: Biopython License Agreement (BSD like, see \url{http://www.biopython.org/DIST/LICENSE}) \\

The Biopython Project is a long-running distributed collaborative effort,
supported by the Open Bioinformatics Foundation, which develops a freely
available Python library for biological computation \cite{AppNote}. This
talk will look ahead to the year to come, and give a summary of the project
news since the 1.74 release in July 2019, and the talk at BOSC 2019.

There have been three releases: Biopython 1.75 (November 2019), Biopython 1.76
(December 2019), and Biopython 1.77 (expected May/June 2020). This year saw
the adoption of the \texttt{black} Python coding style, our final release to
support Python 2, and substantial code cleanup to focus on Python 3 only.

In 2017 we started a re-licensing plan, to transition away
from our liberal but unique \emph{Biopython License Agreement} to the similar
but very widely used \emph{3-Clause BSD License}. We are reviewing the code
base authorship file-by-file, to gradually dual license the entire project.
All new contributions are dual licensed, and currently over $75\%$ of the
Python and C files in the main library have been dual licensed.

Another important effort had been improving the unit test coverage, which can
be viewed at \href{https://codecov.io/github/biopython/biopython/}{CodeCov.io}.
Sadly, this has stalled at about $85\%$ (excluding online tests) for some time.

We are using GitHub-integrated continuous integration testing on Linux (using
\href{https://travis-ci.org/biopython/biopython/builds}{TravisCI}) and Windows
(using \href{https://ci.appveyor.com/project/biopython/biopython/history}{AppVeyor}),
including enforcing the Python PEP8, PEP257 and black coding style guidelines.
We recommend a simple \texttt{git pre-commit} hook using \textit{flake8} for
our contributors, which aims to reduce the human time costs in writing
compliant code.

Finally, since our last update talk in July 2019, Biopython has had 37 named
contributors including 15 newcomers. This reflects our policy of trying to
encourage even small contributions. Our total named contributor count is now
at 275 since the project began, over twenty year ago.

\begin{thebibliography}{}

\bibitem[Cock {\it et al}., 2009]{AppNote}Cock, P.J.A., Antao, T., Chang, J.T., Chapman, B.A., Cox, C.J., Dalke, A., Friedberg, I., Hamelryck, T., Kauff, F., Wilczynski, B., de Hoon, M.J. (2009) Biopython: freely available Python tools for computational molecular biology and bioinformatics. {\it Bioinformatics} {\bf 25}(11) 1422-3. \href{http://dx.doi.org/10.1093/bioinformatics/btp163}{doi:10.1093/bioinformatics/btp163}

\end{thebibliography}

\end{document}
